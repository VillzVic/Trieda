\chapter{Introduction}
\label{chap:intro}

%%%%%%%%%%%%%%%%%%%%%%%%%%%%%%%%%%%%%%%%%%%%%%%%%%%%%%%%%%%%%%%%%%%%%%%%%%%%%%%%%%

Here the ideas discussed in this dissertation are contextualized. Giving this, a brief introduction, some definitions and main variants to the Timetabling Problem (TTP) are provided.

% ------------------------------------------------------------
\section{Introducing the basic timetabling problem}

Timetabling problems are a kind of practical scheduling challenge faced by many organizations and important for their daily operations to succeed. Application areas vary widely from sport event timetabling to nurse scheduling, lectures scheduling and transportation timetabling. The general problem can be described as needing to satisfy a set of demands through the scheduling of meetings between those who demand and the available resources. The resulting timetables must be feasible and acceptable to all people and resources involved.

This work is about timetabling in the school environment, which is where the original problem was set.

In school timetabling, a given number of lectures, involving students, teachers and classrooms, should be schedule over a fixed period of time (typically a week), while usually having to satisfy a set of additional constraints of various types. For an actual and applied purpose, besides mandatory operational constraints, such as avoiding resources overbooking, it is essential to consider several solution quality constraints, which can be related to institution, pedagogical and personal needs.

Because schools differ a lot in their educational policies, both when considering the same country and especially more between distinct countries, the criteria for quality and even timetable feasibility definition also depends on each educational system. Here the Brazilian school educational system is embraced. Studies based on the German school system can be found at \cite{Michael2002}, on the Greek system at \cite{Birbas1997} and \cite{Birbas2009}, on Brazilian system at \cite{Haroldo2012}. An overview of school timetabling research is provided by \cite{SchoolOverview2010}, describing the most common operational and quality requirements and the different solving methods that have been used.

University timetabling problems have also been very studied and have many similarities with the school version. Studies based on a Turkish university system can be found at \cite{Guenalay2006} and on a Greek university system at \cite{Daskalaki2005}. For more complete approaches, one can check at \cite{Carter2001} for a Canada university system and at \cite{Murray2007} for an American university system.

The problem is known to be a hard problem and therefore many researchers have shown an interest in it since the early 1960's. Schaerf \cite{Schaerf99} showed that the school timetabling problem is NP-complete and that exact optimal solution could be obtained only for small size problems. Furthermore, enhancing a model with real-world practical and quality requirements increases even more the complexity of the problem. For instance, considering the compactness requirement for students timetables (see \ref{constrStudentGap}), \cite{Asratian2002} shows that deciding whether a timetable that guarantees compactness for the students exits is NP-complete by itself.



% ------------------------------------------------------------
\section{Brazilian school timetabling problem}

In Brazil, basic educational system is split in 3 stages:
\begin{enumerate}
\item Fundamental Education I
\item Fundamental Education II
\item Upper secondary education
\end{enumerate}

Fundamental Education is mandatory for children ages $6-14$. There are 9 \q{years} (as opposed to the former 8 \q{grades}). The current \q{First Year} broadly corresponds to the former Pre-School last year of private institutions, and its aim is to achieve literacy. Generally speaking, the only prerequisite for enrolling in first year is that a child should be 6 years old.

The Federal Council of Education establishes a core curriculum consisting of Portuguese language, history, geography, science, mathematics, arts and physical education (for years 2, 3, 4 and 5). As for years 6, 7, 8 and 9, one or two foreign languages are also compulsory (usually an English and optional language).

Each education system supplements this core curriculum with a diversified curriculum defined by the needs of the region and the abilities of individual students.

Fundamental Education is divided in two stages, called \q{Ensino Fundamental I} (years $1-5$) and \q{Ensino Fundamental II} (years $6-9$). During \q{Ensino Fundamental I} each group of students is usually assisted by a single teacher. As for \q{Ensino Fundamental II}, there are as many teachers as subjects.

Public fundamental schools are funded by municipal and state governments. The education is similar to the British.

Students must have finished their Fundamental education before they are allowed to enroll in Secondary education (\q{Ensino M\'{e}dio}), which takes 3 years. Secondary education core curriculum comprises Portuguese (including Portuguese language, Brazilian and Portuguese literature), foreign language (usually English, also Spanish and very rarely French today), History, Geography, Mathematics, Physics, Chemistry and Biology. Philosophy and Sociology, which were banned during the military dictatorship (1964 to 1985), became compulsory again. The courses provided during this period are essentially designed to allow a young person to enter into an university. A minority of schools provides professional training along with mainstream secondary education. Professional training courses usually last 2 years and can be taken during the 2nd and 3rd years of Secondary education.

Public middle schools are provided by state governments.

Once a student has successfully completed secondary education, they may continue their studies at a public or private university. To enter a public university, students must sit an entrance exam, known as \q{vestibular}. Entrance exams to a private university are often little more than a formality and, as a consequence, public university degrees are usually valued much more highly than those from private institutions.

The normal practice in Brazilian schools, both public and private, is to mix students of all study directions together in the same class. Due to the entrance exam to universities, it is common though at private schools for students at last year of secondary education to have extra specialized classes, which depend on their choice for undergraduation course.

Data from \cite{WikiEducationInBrazil}.


% ------------------------------------------------------------
\section{Practical and realistic formulation}

\mynotes{Purdue University, \cite{Murray2007}}
It is a widely studied area and many potentially useful algorithms have been offered for solving the university course timetabling problem. Unfortunately, much of the work in this area has been conducted using artificial data sets or based on actual problems that have been greatly simplified. Methods developed have also rarely been extended to the solution of actual university problems of any large scale, instead most were restricted to a single department or school.

The major differences between many of the problems studied and their real life counterparts are the additional complexity imposed by course structures, the variety of constraints imposed, and the distributed responsibility for information needed to solve such problems at a university-wide level.

As pointed out in \cite{Murray2007}, the biggest obstacle to solving actual university course timetabling problems is that the complexity can increase considerably beyond that represented in standard formulations of the problem. As the complexity increases, it is easy to be caught in the dual bind that the problem is both more challenging to develop an effective solution approach for, and this approach is less likely to be usable on other university timetabling problems.

As observed by Carter \cite{Carter2001}, there are few published papers that described actual implementations of course timetabling. From what we know, the first example of an automated and integrated course timetabling and student scheduling system across a institution was developed for the University of Waterloo between 1979 and 1985. Although the system was implemented 29 years ago, much of the discussion is still very relevant. By 2001, Carter expressed at \cite{Carter2001} his belief that there was still no system for solving the large-scale course timetabling problem with such level of mathematical sophistication.

By 2006, Guenalay and Sahin introduced at \cite{Guenalay2006} a Decision Support System (DSS) for the university timetabling that allows the direct involvement of the decision maker. They affirmed that their model was the first one which simultaneously combined such a DSS approach with a goal programming optimization tool.

\fixme{citar software!}


% ------------------------------------------------------------
\section{Main similar formulations and their differences}

For solving the high school timetabling problem, this work is based on an integer programming formulation together with strategies for helping at the convergence in the best solution search process.

Among the existing published works that considered more realistic formulations for high school timetabling problems, we highlight \cite{Birbas2009} and \cite{Birbas2009} as those whose embraced problems were the most similar to the one considered in this work. Several practical issues emphasized by them were also identified during TRIEDA's development and were referenced along this work.

Still, there are differences between their timetabling problems and the one here presented. Following, the major points are listed.
-
-
-

Further differences can be identified along this work.

Because TRIEDA is a commercial software, it has been developed to be as portable and flexible as possible. Particularly, the system embraces most common rules and cases found in Brazilian universities and schools.


% ------------------------------------------------------------
\section{Data accuracy and Politics}

\mynotes{from \cite{Murray2007}}
Timetabling is a resource allocation problem; therefore, at most universities responsibility for constructing the timetable is distributed among the academic units with the faculty, physical facilities, and other resources required for offering instruction. Murray, M\"{u}ller and Rudov\'{a} observe at \cite{Murray2007} that providing support for this distributed responsibility is important because departmental timetablers have a much more intimate knowledge of the needs of the courses offered, the professor who might be able to teach a particular class, and the spaces available for specialized instruction than any database that might be maintained centrally. Maintaining each department's sense of ownership in the timetables that are produced is also an important factor in their acceptance of the solutions produced by an automated timetabling process. They say the process needs to be one that assists them rather than replaces them.

Obtaining coherent and correct data is a particular important and sensitive issue. Because data accuracy depends on each department, it is extremely important that data collection phase of an automating process has a deep involvement of all of those who usually operate in the traditional manual process. Any inaccuracy of data can result in a bad or even non-deployable solution.

\fixme{citar funcionalidade: motivos de nao atendimento}

Murray, M\"{u}ller and Rudov\'{a} tells at \cite{Murray2007} that inequity in the quality of time and room assignments received by different departments and faculty members doomed a previous attempt at automating the timetabling process at Purdue University. A similar situation was faced by TRIEDA in July of 2014 while an attempt of deploying a solution in a Brazilian university.

Automating people assignments is different from automating other process --- the optimization factor is definitely \textbf{not} more important than people satisfaction. Only keeping that in mind it is possible to obtain an actual deployable solution.

As wisely pointed out in \cite{Carter2001}, practical course timetabling is 10\% graph theory, and 90\% politics. Carter tells that when they first began designing the system for University of Waterloo, they were warned: \q{You can not dictate to professors when they will teach courses}. Consequently, they were told that course timetabling could not work. Or, more precisely, they can not assume that the timetable can use a clear slate. There are many possible reasons for this. Part-time professors may be available only on certain days or times. University professors often have other commitments and industrial research projects. Teaching is only a part of their job, and probably less than half of their time is devoted to teaching.

It follows that it is essential to allow professors to restrict their available timetables as much (or as little) as they want.

Guenalay and Sahin described at \cite{Guenalay2006} an approach for university timetabling with a goal programming optimization tool in order to process the instructor teaching time preferences. As they observed, using a multi-objective decision making model to solve the problem, and hence considering instructor preferences by using weights, cannot be considered consistently reliable. Such approach unifies the disparate goals of the model, but depends on very subjective choice of weights.

Professors' availabilities and preferences issue should therefore not be underestimated.


The primary design goal is to assist academic timetablers with the problem of building a good timetable, not necessarily finding a true optimal solution.


% ------------------------------------------------------------
\section{Timetabling Conferences}




% ------------------------------------------------------------

This is a test reference to \cite{Carter2001}.    % Waterloo, Canada
This is a test reference to \cite{Murray2007}.    % Purdue (unitime), USA
This is a test reference to \cite{Unitime}.       % website
This is a test reference to \cite{DSS}.           % website
This is a test reference to \cite{Guenalay2006}.  % Bilkent, Turkey, (DSS)
This is a test reference to \cite{Schaerf99}.  		% Schaerf99
This is a test reference to \cite{Michael2002}.   % Michael2002
This is a test reference to \cite{SchoolOverview2010}.   % SchoolOverview2010
This is a test reference to \cite{Birbas2009}.    % Birbas2009
This is a test reference to \cite{Mimosasoftware}.
This is a test reference to \cite{Patat}.
This is a test reference to \cite{Hstt}.				  % Benchmarking project for highschool
This is a test reference to \cite{ITC2007}.				% International Timetabling Competition
This is a test reference to \cite{Ectt}.					% Udine, Italy
This is a test reference to \cite{Udine}.
This is a test reference to \cite{Watt}.					% Working group on automated timetabling




%%%%%%%%%%%%%%%%%%%%%%%%%%%%%%%%%%%%%%%%%%%%%%%%%%%%%%%%%%%%%%%%%%%%%%%%%%%%%%%%%%

\section{Dissertation Outline}
This work is organized as follows: 
\begin{itemize}
	\item Chapter \ref{chap:timetabling} introduces Trieda's timetabling problem.
	\item Chapter \ref{chap:mipformulation} shows the mathematical formulation for the problem.
	\item Chapter \ref{chap:strategies} presents different approaches and strategies for solving the MIP formulation.
	\item Chapter \ref{chap:constranalysis} analyzes the impact of some requirements in the MIP solving.
	\item Chapter \ref{chap:converting} presents an attempt to converting a XHSTT format into a Trieda's problem format.
	\item Chapter \ref{chap:conclusions} contains the conclusions of this work.
\end{itemize}