\chapter{Computational Experiments}
\label{chap:experiments}


This chapter describes the main computational experiments made in this work. First, general computational aspects are introduced; then different scenarios for different schools are described; and finally results are presented and evaluated.


%%%%%%%%%%%%%%%%%%%%%%%%%%%%%%%%%%%%%%%%%%%%%%%%%%%%%%%%%%%%%%%%%%%%%%%%%%%%%%%%%%
\section{General}

The implementation of the solver was done in C++, compiled with MS Visual Studio 2010 environment, using Microsoft Windows 7, 64 bits.

All the linear integer programs were solved using the generic MIP-Solver Gurobi 6.0. 

Computational experiments were executed on 3.2GHz Intel Core i7 computer with 32 GB of RAM.
\fixme{maximo de RAM usado observado foi uns 10gb}


%%%%%%%%%%%%%%%%%%%%%%%%%%%%%%%%%%%%%%%%%%%%%%%%%%%%%%%%%%%%%%%%%%%%%%%%%%%%%%%%%%
\section{Scenarios}

Several computational experiments have been performed for real scenarios of Brazilian schools. Those considered the most significant are following detailed.




%%%%%%%%%%%%%%%%%%%%%%%%%%%%%%%%%%%%%%%%%%%%%%%%%%%%%%%%%%%%%%%%%%%%%%%%%%%%%%%%%%
\section{Results}

\subsection{Model features}

Following, model features for each instance are listed, which include the number of variables and restrictions created, detailed by type, and the reduced size of the model after applying Gurobi pre-solve method.


\subsection{Solver performance}

Whenever the solver is executed, analyzing performance and consequently the generated solution implies in analyzing every solving phase. This means checking for every step the running time, the optimization stopping condition that was reached, the best solution value of the linear program, the optimality gap and also how the solving process converged during the polishing method.


\subsection{Solution quality}

Evaluating quality of a final solution is not a trivial task, specially because there are multiple and conflicting goals involved. Often analyzing and understanding solutions require a deeper analysis of the problem data itself.

Following several solution quality indicators are listed and detailed for the experiments made.


