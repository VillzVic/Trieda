\chapter{Conclusions}
\label{chap:conclusions}




%%%%%%%%%%%%%%%%%%%%%%%%%%%%%%%%%%%%%%%%%%%%%%%%%%%%%%%%%%%%%%%%%%%%%%%%%%%%%%%%%%
\section{Best approaches}



%%%%%%%%%%%%%%%%%%%%%%%%%%%%%%%%%%%%%%%%%%%%%%%%%%%%%%%%%%%%%%%%%%%%%%%%%%%%%%%%%%
\section{What did not work well}

Several practical issues emphasized by other authors were also identified during TRIEDA's development and were referenced along this work. Probably some challenges and problems faced both at solving methods development and deployment phase could have been avoided or at least minimized if we had a previous knowledge of these works.


A first approach divided the problem in two modules, which were called 'tactical' and 'operational' module.



%%%%%%%%%%%%%%%%%%%%%%%%%%%%%%%%%%%%%%%%%%%%%%%%%%%%%%%%%%%%%%%%%%%%%%%%%%%%%%%%%%
\section{Future Work}

There is a common situation in schools that was not handle at this work. Although it is generally an exception, it is common that for some few subjects a class is split. Usually it is the case of language courses, where the student have to choose for a foreign language among some options. In this work we handle situations where different classes are merged, but the opposite case is still to be done.



%%%%%%%%%%%%%%%%%%%%%%%%%%%%%%%%%%%%%%%%%%%%%%%%%%%%%%%%%%%%%%%%%%%%%%%%%%%%%%%%%%