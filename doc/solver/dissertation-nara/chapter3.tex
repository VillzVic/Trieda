\chapter{Mathematical Formulation}
\label{chap:mipformulation}


%%%%%%%%%%%%%%%%%%%%%%%%%%%%%%%%%%%%%%%%%%%%%%%%%%%%%%%%%%%%%%%%%%%%%%%%%%%%%%%%%%

This chapter describes a method for solving the problem which is purely based on mixed integer programming (MIP).

This approach divides the problem in two modules, which were called 'tactical' and 'operational' module.


%%%%%%%%%%%%%%%%%%%%%%%%%%%%%%%%%%%%%%%%%%%%%%%%%%%%%%%%%%%%%%%%%%%%%%%%%%%%%%%%%%
\section{Integer programming}


%%%%%%%%%%%%%%%%%%%%%%%%%%%%%%%
\subsection{Mixed integer program}

Introduces MIP concept and general formulation.


%%%%%%%%%%%%%%%%%%%%%%%%%%%%%%%
\subsection{Linear relaxation}


%%%%%%%%%%%%%%%%%%%%%%%%%%%%%%%
\subsection{Integrality gap}

Introduces integrality gap concept.

As in all combinatorial scheduling models, the problem grows more complex as the number of side constraints increases.\fixme{Colocar essa frase no lugar adequado!}


%%%%%%%%%%%%%%%%%%%%%%%%%%%%%%%
\subsection{Branch and Bound}

Introduces branch and bound concept.


%%%%%%%%%%%%%%%%%%%%%%%%%%%%%%%
\subsection{MIP Solver}

Introduces CPLEX and Gurobi.


